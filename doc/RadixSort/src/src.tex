\section{Описание}

Ключевые моменты использованного алгоритма поразрядной сортировки заключаются в использовании кодов таблицы ASCII вместо строк-ключей, сдвиге индексов в зависимости от типа обрабатываемого символа, чтобы использовать промежуточный контейнер эффективнее, заполняя его с нулевого элемента.

\pagebreak

\section{Исходный код}

Для каждой строки входных данных создаём экземпляры структур, имеющих в качестве полей ключ и значение, значением выступает целое число, являющееся индексом элемента вектора, хранящего в себе все значения из входных данных. После считывания данных переходим к их сортировке, состоящей из двух этапов, заключённых в цикле, обходящем ключи по всем разрядам с конца: подсчёт количества элементов, заканчивающихся на рассматриваемый на данной итерации символ, сбор промежуточного (на последней итерации итогового) вектора. После завершения сортировки вектора, происходит вывод пар ключей и значений, для удобства был перегружен оператор потокового вывода. 

\begin{lstlisting}[language=C]

#include <iostream>
#include <vector>
#include <string>

struct TNumber
{
	std::string key;
	int value;
};

std::ostream& operator<<(std::ostream &strm, const TNumber &elem)
{
	strm << (char)(elem.key[0]) << ' ' \
	<< (char)(elem.key[1]) \
	<< (char)(elem.key[2]) \
	<< (char)(elem.key[3]) << ' ' \
	<< (char)(elem.key[4]) \
	<< (char)(elem.key[5]);
	
	return strm;
}

void RadixSort(std::vector<TNumber> &data) 
{
	std::vector<TNumber> res(data.size());
	
	for (int currentDigit = 5; currentDigit >= 0; currentDigit--) 
	{
		std::vector<unsigned> count(26);
		
		for (auto &elem: data) 
		{
			char tmp = elem.key[currentDigit];
			
			if (isalpha(tmp)) {
				++count[tmp - 'A'];
			}
			else if (isdigit(tmp)) {
				++count[tmp - '0'];
			}
		}
		
		for (int i = 1; i < count.size(); i++) {
			count[i] = count[i - 1] + count[i];
		}
		
		for (int i = (int) data.size() - 1; i >= 0; i--) 
		{
			char tmp = data[i].key[currentDigit];
			
			if (isalpha(tmp)) {
				res[--count[tmp - 'A']] = data[i];
			} 
			else if (isdigit(tmp)) {
				res[--count[tmp - '0']] = data[i];
			}
		}
		
		data = res;
	}
}

int main()
{
	std::ios_base::sync_with_stdio(false);
	std::cin.tie(NULL);
	std::cout.tie(NULL);
	
	std::string inputStr;
	std::vector<TNumber> data;
	
	std::vector<std::string> values;
	int idx = 0;
	
	std::string first, second, third, value;
	
	while (std::cin >> first)
	{
		TNumber elem;
		
		std::cin >> second >> third;
		elem.key = first + second + third;
		
		std::cin >> value;
		elem.value = idx;
		
		values.push_back(value);
		++idx;
		
		data.push_back(elem);
	}
	
	if (data.size()){
		RadixSort(data);
	}
	
	for (const TNumber & elem : data){
		std::cout << elem << '\t' << values[elem.value] << '\n';
	}
	
	return 0;
}
	
\end{lstlisting}



\section{Консоль}
\begin{alltt}
[axr@pekarnya trash]$ g++ -Wall -o LSD LSD_alter.cpp
[axr@pekarnya trash]$ ./LSD
	A 000 AA	xGfxrxGGxrxMMMMfrrrG
	Z 999 ZZ	xGfxrxGGxrxMMMMfrrr
	A 000 AA	xGfxrxGGxrxMMMMfrr
	Z 999 ZZ	xGfxrxGGxrxMMMMfr
	A 000 AA	xGfxrxGGxrxMMMMfrrrG
	A 000 AA	xGfxrxGGxrxMMMMfrr
	Z 999 ZZ	xGfxrxGGxrxMMMMfrrr
	Z 999 ZZ	xGfxrxGGxrxMMMMfr
	[axr@pekarnya trash]$ 
\end{alltt}
\pagebreak

